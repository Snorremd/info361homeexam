SO what is a robot? In the Oxford Dictionaries a robot is defined as a ``\textit{a machine capable of carrying out a complex series of actions automatically, especially one programmable by a computer [\dots]}'', and further notes that it's origin is ``\textit{from Czech, from robota `forced labour' [\dots]}'' \parencite{OxfordDictionaries2010robot}. This definition combined with the word's origin do signify a certain type of robot, and the word could thus be construed to mean a machine capable of doing complex, but repetitive actions automatically (for instance an industrial robot). Another definition found in the dictionary defines robot as: ``\textit{(especially in science fiction) a machine resembling a human being and able to replicate certain human movements and functions automatically}'' \parencite{OxfordDictionaries2010robot}.

Many different types of robots exist ranging from industrial robot arms for industrial production to more humanoid robots designed to interact with humans. \textcite{Breazeal2004robot} list four paradigms in Human-Robot Interaction:
\begin{itemize}
	\item robot as tool
	\item robot as cyborg extension
	\item robot as avatar
	\item robot as sociable partner.
\end{itemize}
In this paper I will focus primarily on the last category, that is robot as a sociable partner. This is also the kind of robot detailed in the suggested paper: \citetitle{Kahn2008robot}.