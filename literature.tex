\subsection{Social robots}

\textcite{Kahn2008robot} write about how design patterns can benefit the field of HRI. A design pattern is a description of a problem that occurs multiple times in our environment and describes a solution to that problem so that this solution can be used over and over without said solution being the same twice (Alexander, 1979 as cited in \cite{Kahn2008robot}). \textcite{Kahn2008robot} identify four central ideas from works by Alexander: that (design) patterns are sufficiently abstract so that many instances of that pattern can be realized to solve a problem, that patterns can be combined, that organizationally simple patterns can be combined hierarchically to form more complex patterns, and that patterns are patterns of human interaction with the physical and social world. Based on these four central ideas they create eight design patterns for sociality in HRI.

``The Initial Introduction'' is a design pattern than models the initial introduction between two people. \textcite{Kahn2008robot} explain that the pattern use an established verbal and behavioral repertoire such as acknowledging the other person's presence, to ask about the other person's well being (or the like), and lastly to engage in physical acknowledgment in form of a handshake or other similar actions. ``Didactic Communication'' is a pattern in which information is transferred from one individual to another (i.e. one-way communication). In social situations or relationships one often has to physically align oneself with another person or a group of persons. This design pattern the authors call ``In Motion Together''. How this pattern relates to Proxemic Interaction will be explored later in this paper. ``Person Interests  and History'' is exemplified by explaining how the experimenters let Robovie, the robot used in the research, tell the participants about its interest in coral reefs and how it became interested in this topic. The fifth design pattern explored is the pattern ``Recovering From Mistakes'' wherein the situation of mistakes in social interactions are explored. ``Reciprocal Turn-Taking in Game Context'' is a pattern that is used to describe social games where participants take turns. In human interaction physical intimacy, claim \textcite{Kahn2008robot}, plays a large role. ``Physical Intimacy'' is a design pattern that facilitate this type of interaction. The last design pattern detailed by the authors is ``Claiming Unfair Treatment or Wrongful Harms''. This design pattern allows a person (or a robot) to make claims about its moral standing.

We have seen how Alexander's design patterns can be used to describe patterns of social interaction. But what are some of the challenges of designing social robots? In the paper \citetitle{Breazeal2004robot} \textcite{Breazeal2004robot} discuss some of the key challenges of HRI. Initially some key challenges with HRI from a human perspective is detailed such as how HRI compares to other forms of interaction, how to design robots capable of working in teams, if the robots should possess its own personality, how cultural issues impact design and so forth. The human, she argues, is however not the single element of HRI. HRI should also be investigated from the perspective of the robot.

\textcite[][184]{Breazeal2004robot} claim that ``there are many advantages that social cues and skills could offer robots that learn from people.''. ``Learning in the human environment'' is only one of several challenges for social robots mentioned in the paper, but is of particular interest. She lists five key challenges to socially guided learning which will be briefly listed here: knowing what matters, knowing what actions to try, 

In \citetitle{Yanco2004robot} \textcite{Yanco2004robot} provide a taxonomy of Human-Robot Interaction inspired by the fields of Human-computer interaction, Computer-supported cooperative work and robotics. The paper also provides updated categories as well as an added social dimension or nature to the task classification. The paper lists 11 types of taxonomy of which some are of special interest to this paper. The ``TASK'' category denotes the task type of the interaction and a task should be specified at a high level. We will later see how this can be connected to the Activity Theory view of cognition. The ``INTERACTION'' and ``INTERACTION-ROLE'' categories revolve around the distribution of robots and humans, and their roles in HRI. These classification categories can be used when analyzing HRI through the notion of distributed cognition. Lastly the ``PHYSICAL-PROXIMITY'' category defines the level of closeness (collocated versus apart) and various modes of collocation. This last category could be analyzed in terms of Proxemic Interaction \parencite{Yanco2004robot}.